\documentclass[DM,lsstdraft,STR,toc]{lsstdoc}
\usepackage{geometry}
\usepackage{longtable,booktabs}
\usepackage{enumitem}
\usepackage{arydshln}

\input meta.tex

\providecommand{\tightlist}{
  \setlength{\itemsep}{0pt}\setlength{\parskip}{0pt}}

\begin{document}

\def\milestoneName{Camera Data Processing}
\def\milestoneId{LDM-503-07}
\def\product{Data Management}

\setDocCompact{true}

\title[\milestoneId{}~Test Report]{\milestoneId{} (\milestoneName{})~Test Plan and Report}
\setDocRef{\lsstDocType-\lsstDocNum}
\setDocDate{\vcsdate}
\setDocUpstreamLocation{\url{https://github.com/lsst/lsst-texmf/examples}}
\author{ John Swinbank }

\input history_and_info.tex


\setDocAbstract{
This is the test plan and report for \milestoneId{} (\milestoneName{}), an LSST level 2 milestone pertaining to the Data Management Subsystem.
}


\maketitle

\section{Introduction}
\label{sect:intro}


\subsection{Objectives}
\label{sect:objectives}

This test plan demonstrates that the LSST Science Pipelines can
successfully be used to load and perform basic processing on data from
the LSST Camera test systems.\\[2\baselineskip]In particular, it will
demonstrate that:\\[2\baselineskip]

\begin{itemize}
\tightlist
\item
  Data from the Camera test systems has been made available at the LSST
  Data Facility;
\item
  Data from the Camera test systems can be accessed using the ``Data
  Butler'' I/O abstraction, and loaded into the LSST Science Platform
  Notebook Aspect for processing and inspection;
\item
  Basic LSST Science Pipelines Tasks can be used to process and
  manipulate Camera data;
\item
  Camera data can be sent to the Firefly visualization tool for display.
\end{itemize}

Verification that the data processing is ``correct'' falls outside the
scope of this test plan: both Camera data and DM code is evolving
rapidly, and this exercise will not demonstrate that particular
thresholds in terms of data processing fidelity have been reached.
Rather, the focus here is on demonstrating successful integration and
interface compatibility.\\[2\baselineskip]

\subsection{Scope}\label{scope}

The overall strategy for testing and verification with LSST Data
Management is described in \citeds{LDM-503}.\\
Success in this test plan is intended to demonstrate completion of the
milestone LDM-503-07 ``Camera Data Processing''.



\subsection{System Overview}
\label{sect:systemoverview}

This test plan addresses primarily the integration between early data
coming from the LSST Camera and the data access facilities provided by
the LSST Data Management system.\\[2\baselineskip]In the process, it
will exercise:

\begin{itemize}
\tightlist
\item
  The ``Data Butler'' I/O abstraction provided by Data Management;
\item
  The Notebook Aspect of the LSST Science Platform;
\item
  Algorithmic code provided by the LSST Science Pipelines;
\item
  The Firefly image display tool provided by the Science User Interface
  and Tools group.
\end{itemize}

\subsection{Applicable Documents}\label{applicable-documents}

\citeds{LDM-503} Data Management Test Plan\\
\citeds{LDM-151} Data Management Science Pipelines Design\\
\citeds{LDM-152} Data Management Middleware Design\\
\citeds{LDM-542} Science Platform Design


\subsection{Document Overview}
\label{sect:docoverview}

This document was generated from Jira, obtaining the relevant information from the 
\href{https://jira.lsstcorp.org/secure/Tests.jspa#/testPlan/LVV-P16}{LVV-P16}
~Jira Test Plan and related Test Cycles (
  \href{https://jira.lsstcorp.org/secure/Tests.jspa#/testCycle/LVV-C19}{LVV-C19}
).

%The following general sections are completed before the start of the test activity.

Section \ref{sect:intro} provides an overview of the test campaign, the system under test (\product{}), the applicable documentation, and explains how this document is organized.
Section \ref{sect:configuration}  describes the configuration used for this test.
Section \ref{sect:personnel} lists all the individuals involved and describes their roles.
%Section \ref{sect:plannedtestactivities} provides the list of planned test cycles and test cases, including all relevant information that fully describes the test campaign.

Section \ref{sect:overview} provides a summary of the test results, including an overview in Table \ref{table:summary}, an overall assessment statement and suggestions for possible improvements.
Section \ref{sect:detailedtestresults} provides detailed results for each step in each test case.

The current status of test plan LVV-P16 in Jira is Approved.

\subsection{References}
\label{sect:references}
\renewcommand{\refname}{}
\bibliography{lsst,refs,books,refs_ads}
\section{Test Configuration}
\label{sect:configuration}

\subsection{Data Collection}

  Observing is not required for this test campaign.

\subsection{Verification Environment}
\label{sect:hwconf}
  Tests of the Data Butler, the Science Pipelines and the Firefly image
display tool will take place within the Notebook Aspect of the LSST
Science Platform, as deployed at
\url{https://lsst-lspdev.ncsa.illinois.edu/nb} and documented at
https://nb.lsst.io. This provides a flexible and configurable
environment with access to large-capacity filesystems at the LSST Data
Facility.\\[2\baselineskip]Individual tests will be based on specific
machine images provided within the Notebook Aspect, as documented in the
relevant test cases.~





\section{Personnel}
\label{sect:personnel}

The following personnel are involved in this test activity:

\begin{itemize}
\item Test Plan (LVV-P16) owner: John Swinbank
\item Test Cycles:
\begin{itemize}
  \item LVV-C19 owner: 
    Undefined
  \begin{itemize}
    \item Test case LVV-T374 tester: John Swinbank
    \item Test case LVV-T368 tester: John Swinbank
  \end{itemize}
\end{itemize}
\item Additional Test Personnel involved: None
\end{itemize}

\newpage

\section{Overview of the Test Results}
\label{sect:overview}

\subsection{Summary}
\label{sect:summarytable}

\begin{longtable} {p{0.2\textwidth}p{0.2\textwidth}p{0.6\textwidth}}
\toprule
  \multicolumn{3}{c}{ Test Cycle {\bf LVV-C19: LDM-503-07: Camera Data Processing }} \\\hline
  {\bf \footnotesize test case} & {\bf \footnotesize status} & {\bf \footnotesize comment} \\\toprule
    \href{https://jira.lsstcorp.org/secure/Tests.jspa#/testCase/LVV-T368}{LVV-T368} 
    & Pass & 
    \\\hline
    \href{https://jira.lsstcorp.org/secure/Tests.jspa#/testCase/LVV-T374}{LVV-T374} 
    & Pass & 
    \\\hline

\caption{Test Results Summary}
\label{table:summary}
\end{longtable}

\subsection{Overall Assessment}
\label{sect:overallassessment}

The test campaign has been successfully concluded.\\[2\baselineskip]All
steps in the accompanying test scripts have been completed successfully.
One minor typo was discovered in LVV-T368, but since it was a flaw in
the test script rather than the system being tested, it is not regarded
as germane to the success of this test campaign.


\subsection{Recommended Improvements}
\label{sect:recommendations}

The typo in LVV-T368 step 2 (a missing ``\textasciitilde{}/'') should be
resolved.


\newpage
\section{Detailed Test Results}
\label{sect:detailedtestresults}


  \subsection{Test Cycle LVV-C19 }

Open test cycle {\it \href{https://jira.lsstcorp.org/secure/Tests.jspa#/testrun/LVV-C19}{LDM-503-07: Camera Data Processing}} in Jira.

  LDM-503-07: Camera Data Processing\\
  Status: Done

  This test cycle defines tests to be performed in late 2018 to
demonstrate the current state of integration between the Data Management
System and current Camera test datasets.


  \subsubsection{Software Version/Baseline}
    This test will be carried out on whatever versions of the LSST Science
Platform Notebook Aspect and Portal Aspect are installed on
lsst-lspdev.ncsa.illinois.edu at the time of testing. No particular
version requirements are specified on those
tools.\\[2\baselineskip]Pipelines tests will be based on the weekly
w\_2018\_45 version of the codebase.~


  \subsubsection{Configuration}
    No specific configuration is required beyond the availability of the
GPFS filesystem at the LSST Data Facility and access to the Notebook and
Portal Aspects of the Science Platform as described above.


  \subsubsection{Test Cases in LVV-C19 Test Cycle}


    \paragraph{Test Case LVV-T368 }\mbox{}\\

Open  \href{https://jira.lsstcorp.org/secure/Tests.jspa#/testCase/LVV-T368}{\textit{ LVV-T368 } }
test case in Jira.

    This test will check:

\begin{itemize}
\tightlist
\item
  That Camera test data is available for processing in the LSST Data
  Facility, and accessible through the LSST Science Platform;
\item
  That the Data Management I/O abstraction (the ``Data Butler'') can
  load that data into the Science Platform environment;
\item
  That Data Management algorithmic ``tasks'' can be executed to process
  that data;
\item
  That results can be displayed in the Firefly display tool.
\end{itemize}


    {\bf Preconditions}:\\
    Appropriate data --- to include a ``raw'' and a ``bias'' exposure ---
from the Camera test systems must be available in a Butler data
repository on a filesystem accessible to the Notebook Aspect of the
Science Platform.\\[2\baselineskip]For the purposes of the following
discussion, we assume that:\\[2\baselineskip]

\begin{itemize}
\tightlist
\item
  Visit 258334666 from RTM (Raft Tower Module) 007 will be used;
\item
  The data is available in a repository at
  /project/bootcamp/repo\_RTM-007/ on the Data Facility GPFS filesystem
\end{itemize}

In the test script, we refer to ``258334666'' as ``\$VISIT\_ID'' and
``/project/bootcamp/repo\_RTM-007/'' as ``\$REPOSITORY\_PATH''; other
data may be substituted as appropriate.


    Execution status: {\bf Pass }

    Final comment:\\


    Detailed step results:

    \begin{longtable}{p{1cm}p{2cm}p{13cm}}
    \hline
    {Step} & \multicolumn{2}{c}{Description, Results and Status}\\ \hline
      1 & Description &

      \begin{minipage}[t]{13cm}{\footnotesize
      Connect to the Notebook Aspect of the Science Platform following the
instructions at https://nb.lsst.io/. Log in, and ``spawn'' a new machine
with image ``Weekly 2018\_45`` and size ``small''.

      \vspace{\dp0}
      } \end{minipage} \\
      \\ \cdashline{2-3}

      & Expected Result & 

      \begin{minipage}[t]{13cm}{\footnotesize
      The JupyterLab environment appears.

      \vspace{\dp0}
      } \end{minipage} \\
      \\ \cdashline{2-3}

      & \begin{minipage}[t]{2cm}{Actual\\ Result}\end{minipage}   & 
      \begin{minipage}[t]{13cm}{\footnotesize
      \begin{itemize}
\tightlist
\item
  The test was carried out at the University of Washington, which has a
  Data Facility firewall exemption, so instructions at
  https://nb.lsst.io/getting-started/logging-in.html describing
  installing, configuring and connecting with a VPN client were skipped.
\item
  The ``Weekly 2018\_45'' image was not listed --- only the two most
  recent weeklies are called out explicitly. However, ``w201845'' was
  chosen as a close analog.
\item
  The JupyterLab environment appeared as expected.
\end{itemize}

      \vspace{\dp0}
      } \end{minipage} \\
      \\ \cdashline{2-3}

      & Status          & Pass \\ \hline

      2 & Description &

      \begin{minipage}[t]{13cm}{\footnotesize
      Create a terminal session. Use it to set up the LSST tools, then
download and build version 5c12b06e6 of
obs\_lsst:\\[2\baselineskip]\hspace*{0.333em} ~\$ source
/opt/lsst/software/stack/loadLSST.bash\\
\hspace*{0.333em} ~\$ setup lsst\_distrib\\
\hspace*{0.333em} ~\$ git clone https://github.com/lsst/obs\_lsst.git\\
\hspace*{0.333em} ~\$ cd obs\_lsst\\
\hspace*{0.333em} ~\$ git checkout 5c12b06e6\\
\hspace*{0.333em} ~\$ setup -k -r .\\
\hspace*{0.333em} ~\$ scons\\[2\baselineskip]Arrange for obs\_lsst to
automatically be added to the environment when starting a new
notebook:\\[2\baselineskip]\hspace*{0.333em} ~\$ echo ``setup -j -r
\textasciitilde{}/obs\_lsst'' \textgreater{}\textgreater{}
notebooks/.user\_setups\\[2\baselineskip]Exit the terminal.

      \vspace{\dp0}
      } \end{minipage} \\
      \\ \cdashline{2-3}

      & Expected Result & 

      \begin{minipage}[t]{13cm}{\footnotesize
      No errors are seen during execution of the provided commands.

      \vspace{\dp0}
      } \end{minipage} \\
      \\ \cdashline{2-3}

      & \begin{minipage}[t]{2cm}{Actual\\ Result}\end{minipage}   & 
      \begin{minipage}[t]{13cm}{\footnotesize
      No expected result is provided.\\[2\baselineskip]Executing the final
command returned:\\[2\baselineskip]\hspace*{0.333em} ~bash:
notebooks/.user\_setups: No such file or directory\\[2\baselineskip]This
is an obvious typo in the test script. Correcting the command
to\\[2\baselineskip]\hspace*{0.333em} echo ``setup -j -r
\textasciitilde{}/obs\_lsst'' \textgreater{}\textgreater{}
\textasciitilde{}/notebooks/.user\_setups\\[2\baselineskip]caused it to
execute successfully.

      \vspace{\dp0}
      } \end{minipage} \\
      \\ \cdashline{2-3}

      & Status          & Pass \\ \hline

      3 & Description &

      \begin{minipage}[t]{13cm}{\footnotesize
      Create a new ``LSST'' notebook.\\[2\baselineskip]Import the standard
libraries required for the rest of this
test:\\[2\baselineskip]\hspace*{0.333em} ~import os\\
\hspace*{0.333em} ~import lsst.afw.display as afwDisplay\\
\hspace*{0.333em} ~from lsst.daf.persistence import Butler\\
\hspace*{0.333em} ~from lsst.ip.isr import IsrTask\\
\hspace*{0.333em} ~from firefly\_client import FireflyClient\\
\hspace*{0.333em} ~from IPython.display import
IFrame\\[2\baselineskip]and execute the cell.

      \vspace{\dp0}
      } \end{minipage} \\
      \\ \cdashline{2-3}

      & Expected Result & 

      \begin{minipage}[t]{13cm}{\footnotesize
      Nothing is printed.

      \vspace{\dp0}
      } \end{minipage} \\
      \\ \cdashline{2-3}

      & \begin{minipage}[t]{2cm}{Actual\\ Result}\end{minipage}   & 
      \begin{minipage}[t]{13cm}{\footnotesize
      The imports completed successfully.

      \vspace{\dp0}
      } \end{minipage} \\
      \\ \cdashline{2-3}

      & Status          & Pass \\ \hline

      4 & Description &

      \begin{minipage}[t]{13cm}{\footnotesize
      Create a Data Butler client, and use it to retrieve the data which will
be used for this test.\\[2\baselineskip]\hspace*{0.333em} ~butler =
Butler(\$REPOSITORY\_PATH)\\
\hspace*{0.333em} ~raw = butler.get(``raw'', visit=\$VISIT\_ID,
detector=2)\\
\hspace*{0.333em} ~bias = butler.get(``bias'', visit=\$VISIT\_ID,
detector=2)

      \vspace{\dp0}
      } \end{minipage} \\
      \\ \cdashline{2-3}

      & Expected Result & 

      \begin{minipage}[t]{13cm}{\footnotesize
      Nothing is printed.

      \vspace{\dp0}
      } \end{minipage} \\
      \\ \cdashline{2-3}

      & \begin{minipage}[t]{2cm}{Actual\\ Result}\end{minipage}   & 
      \begin{minipage}[t]{13cm}{\footnotesize
      The commands were executed successfully.\\[2\baselineskip]

      \vspace{\dp0}
      } \end{minipage} \\
      \\ \cdashline{2-3}

      & Status          & Pass \\ \hline

      5 & Description &

      \begin{minipage}[t]{13cm}{\footnotesize
      Initialize the Firefly display
system:\\[2\baselineskip]\hspace*{0.333em} ~my\_channel =
`\{\}\_test\_channel'.format(os.environ{[}'USER'{]})\\
\hspace*{0.333em} ~server = `https://lsst-lspdev.ncsa.illinois.edu'\\
\hspace*{0.333em}
~ff='\{\}/firefly/slate.html?\_\_wsch=\{\}'.format(server,
my\_channel)\\
\hspace*{0.333em} ~IFrame(ff,800,600)\\
\hspace*{0.333em} ~afwDisplay.setDefaultBackend('firefly')\\
\hspace*{0.333em} ~afw\_display = afwDisplay.getDisplay(frame=1,\\
\hspace*{0.333em} ~ ~ ~ ~ ~ ~ ~ ~ ~ ~ ~ ~ ~ ~ ~ ~ ~ ~
~name=my\_channel)\\[2\baselineskip]Click on the link provided after
executing the above.

      \vspace{\dp0}
      } \end{minipage} \\
      \\ \cdashline{2-3}

      & Expected Result & 

      \begin{minipage}[t]{13cm}{\footnotesize
      A Firefly window is shown.

      \vspace{\dp0}
      } \end{minipage} \\
      \\ \cdashline{2-3}

      & \begin{minipage}[t]{2cm}{Actual\\ Result}\end{minipage}   & 
      \begin{minipage}[t]{13cm}{\footnotesize
      Firefly window appears.

      \vspace{\dp0}
      } \end{minipage} \\
      \\ \cdashline{2-3}

      & Status          & Pass \\ \hline

      6 & Description &

      \begin{minipage}[t]{13cm}{\footnotesize
      Display the raw image data in the Firefly
window:\\[2\baselineskip]\hspace*{0.333em} afw\_display.mtv(raw)

      \vspace{\dp0}
      } \end{minipage} \\
      \\ \cdashline{2-3}

      & Expected Result & 

      \begin{minipage}[t]{13cm}{\footnotesize
      Raw image data is displayed.

      \vspace{\dp0}
      } \end{minipage} \\
      \\ \cdashline{2-3}

      & \begin{minipage}[t]{2cm}{Actual\\ Result}\end{minipage}   & 
      \begin{minipage}[t]{13cm}{\footnotesize
      Image data was seen in the Firefly window.

      \vspace{\dp0}
      } \end{minipage} \\
      \\ \cdashline{2-3}

      & Status          & Pass \\ \hline

      7 & Description &

      \begin{minipage}[t]{13cm}{\footnotesize
      Configure and run an Instrument Signature Removal (ISR) task on the raw
data. Most corrections are disabled for simplicity. but the bias frame
is applied.\\
\hspace*{0.333em}

\begin{verbatim}
   isr_config = IsrTask.ConfigClass()
   isr_config.doDark=False
   isr_config.doFlat=False
   isr_config.doFringe=False
   isr_config.doDefect=False
   isr_config.doAddDistortionModel=False
   isr_config.doLinearize=False
   isr = IsrTask(config=isr_config)
   result = isr.run(raw, bias=bias)
\end{verbatim}

      \vspace{\dp0}
      } \end{minipage} \\
      \\ \cdashline{2-3}

      & Expected Result & 

      \begin{minipage}[t]{13cm}{\footnotesize
      Nothing is printed.

      \vspace{\dp0}
      } \end{minipage} \\
      \\ \cdashline{2-3}

      & \begin{minipage}[t]{2cm}{Actual\\ Result}\end{minipage}   & 
      \begin{minipage}[t]{13cm}{\footnotesize
      Commands successfully executed.

      \vspace{\dp0}
      } \end{minipage} \\
      \\ \cdashline{2-3}

      & Status          & Pass \\ \hline

      8 & Description &

      \begin{minipage}[t]{13cm}{\footnotesize
      Display the corrected image data in the Firefly
window:\\[2\baselineskip]\hspace*{0.333em}
~afw\_display.mtv(result.exposure)

      \vspace{\dp0}
      } \end{minipage} \\
      \\ \cdashline{2-3}

      & Expected Result & 

      \begin{minipage}[t]{13cm}{\footnotesize
      Processed (trimmed, bias-subtracted) image data is displayed.

      \vspace{\dp0}
      } \end{minipage} \\
      \\ \cdashline{2-3}

      & \begin{minipage}[t]{2cm}{Actual\\ Result}\end{minipage}   & 
      \begin{minipage}[t]{13cm}{\footnotesize
      Processed data was displayed in the Firefly window.

      \vspace{\dp0}
      } \end{minipage} \\
      \\ \cdashline{2-3}

      & Status          & Pass \\ \hline

    \end{longtable}


    \paragraph{Test Case LVV-T374 }\mbox{}\\

Open  \href{https://jira.lsstcorp.org/secure/Tests.jspa#/testCase/LVV-T374}{\textit{ LVV-T374 } }
test case in Jira.

    This test will check:

\begin{itemize}
\tightlist
\item
  That raw Camera test data is available on a filesystem in the LSST
  Data Facility;
\item
  That raw Camera test data can be ingested and made available through
  the Data Management I/O abstraction (the ``Data Butler'').
\end{itemize}


    {\bf Preconditions}:\\
    Appropriate raw data from Camera test systems must be available on a
filesystem within the LSST Data Facility. This test data is assumed to
include visit 258334666 (hereafter referred to as \$VISIT\_ID) from RTM
(Raft Tower Module) 007 for the purposes of this test, but other,
equivalent, may be substituted.\\[2\baselineskip]At time of writing,
suitable data may be found on the GPFS filesystem at
/project/bootcamp/data/LCA-11021\_RTM-007/7086/fe55\_raft\_acq/v0/44981.
In future, as data transport procedures to the Data Facility become more
streamlined and formalised, this data may be moved elsewhere or made
available through some other system. Throughout the test script, we use
the string ``\$INPUT\_DATA\_DIR'' as an alias for
``/project/bootcamp/data/LCA-11021\_RTM-007/7086/fe55\_raft\_acq/v0/44981'',
or wherever this data has been moved to.\\[2\baselineskip]


    Execution status: {\bf Pass }

    Final comment:\\


    Detailed step results:

    \begin{longtable}{p{1cm}p{2cm}p{13cm}}
    \hline
    {Step} & \multicolumn{2}{c}{Description, Results and Status}\\ \hline
      1 & Description &

      \begin{minipage}[t]{13cm}{\footnotesize
      Connect to the Notebook Aspect of the Science Platform following the
instructions at https://nb.lsst.io/. Log in, and ``spawn'' a new machine
with image ``Weekly 2018\_45`` and size ``large''.

      \vspace{\dp0}
      } \end{minipage} \\
      \\ \cdashline{2-3}

      & Expected Result & 

      \begin{minipage}[t]{13cm}{\footnotesize
      The JupyterLab environment appears.

      \vspace{\dp0}
      } \end{minipage} \\
      \\ \cdashline{2-3}

      & \begin{minipage}[t]{2cm}{Actual\\ Result}\end{minipage}   & 
      \begin{minipage}[t]{13cm}{\footnotesize
      \begin{itemize}
\tightlist
\item
  The test was carried out at the University of Washington, which has a
  Data Facility firewall exemption, so instructions at
  https://nb.lsst.io/getting-started/logging-in.html describing
  installing, configuring and connecting with a VPN client were skipped.
\item
  The ``Weekly 2018\_45'' image was not listed --- only the two most
  recent weeklies are called out explicitly. However, ``w201845'' was
  chosen as a close analog.
\item
  The JupyterLab environment appeared as expected.
\end{itemize}

      \vspace{\dp0}
      } \end{minipage} \\
      \\ \cdashline{2-3}

      & Status          & Pass \\ \hline

      2 & Description &

      \begin{minipage}[t]{13cm}{\footnotesize
      Create a terminal session. Use it to set up the LSST tools, then
download and build version 5c12b06e6 of
obs\_lsst:\\[2\baselineskip]\hspace*{0.333em} ~\$ source
/opt/lsst/software/stack/loadLSST.bash\\
\hspace*{0.333em} ~\$ setup lsst\_distrib\\
\hspace*{0.333em} ~\$ git clone https://github.com/lsst/obs\_lsst.git\\
\hspace*{0.333em} ~\$ cd obs\_lsst\\
\hspace*{0.333em} ~\$ git checkout 5c12b06e6\\
\hspace*{0.333em} ~\$ setup -k -r .\\
\hspace*{0.333em} ~\$ scons

      \vspace{\dp0}
      } \end{minipage} \\
      \\ \cdashline{2-3}

      & Expected Result & 

      \begin{minipage}[t]{13cm}{\footnotesize
      No errors are seen during execution of the provided commands.

      \vspace{\dp0}
      } \end{minipage} \\
      \\ \cdashline{2-3}

      & \begin{minipage}[t]{2cm}{Actual\\ Result}\end{minipage}   & 
      \begin{minipage}[t]{13cm}{\footnotesize
      Commands were executed with no errors. Last line printed was ``scons:
done building targets''.

      \vspace{\dp0}
      } \end{minipage} \\
      \\ \cdashline{2-3}

      & Status          & Pass \\ \hline

      3 & Description &

      \begin{minipage}[t]{13cm}{\footnotesize
      Ingest RTM-007 test data by executing the following
commands:\\[2\baselineskip]\hspace*{0.333em}
~OUTPUT\_REPO\_DIR=\$OUTPUT\_DATA\_DIR\\
\hspace*{0.333em} ~INPUT\_DATA\_DIR=\$INPUT\_DATA\_DIR\\
\hspace*{0.333em} ~mkdir -p \$OUTPUT\_REPO\_DIR\\
\hspace*{0.333em} ~echo ``lsst.obs.lsst.ts8.Ts8Mapper'' \textgreater{}
\$OUTPUT\_REPO\_DIR/\_mapper\\
\hspace*{0.333em} ~ingestImages.py \$OUTPUT\_REPO\_DIR
\$INPUT\_DATA\_DIR/*/*.fits\\
\hspace*{0.333em} ~constructBias.py \$OUTPUT\_REPO\_DIR --rerun calibs
--id imageType=BIAS --batch-type smp --cores 4\\
\hspace*{0.333em} ~ingestCalibs.py \$OUTPUT\_REPO\_DIR --calibType bias
\$OUTPUT\_REPO\_DIR/rerun/calibs/bias/*/*.fits --validity 9999 --output
\$OUTPUT\_REPO\_DIR/CALIB
--mode=link\\[2\baselineskip]Where:\\[2\baselineskip]\hspace*{0.333em}
~\$OUTPUT\_DATA\_DIR is some location on shared storage to which the
user has write permission;\\
\hspace*{0.333em} ~\$INPUT\_DATA\_DIR is defined in the test case
description.

      \vspace{\dp0}
      } \end{minipage} \\
      \\ \cdashline{2-3}

      & Expected Result & 

      \begin{minipage}[t]{13cm}{\footnotesize
      Many status messages are logged to screen, and the command exits with
status 0.

      \vspace{\dp0}
      } \end{minipage} \\
      \\ \cdashline{2-3}

      & \begin{minipage}[t]{2cm}{Actual\\ Result}\end{minipage}   & 
      \begin{minipage}[t]{13cm}{\footnotesize
      \begin{itemize}
\tightlist
\item
  A number of ``WARN'' messages were logged during constructBias.py.
  These are not important for the correct execution of the test, but
  should be suppressed or included in the test description.
\item
  All commands executed successfully.
\end{itemize}

      \vspace{\dp0}
      } \end{minipage} \\
      \\ \cdashline{2-3}

      & Status          & Pass \\ \hline

      4 & Description &

      \begin{minipage}[t]{13cm}{\footnotesize
      Demonstrate that raw and bias data for visit \$VISIT\_ID have been made
available in the repository. Load a Python interpreter (run ``python'')
and execute the following:\\[2\baselineskip]\hspace*{0.333em} ~from
lsst.daf.persistence import Butler\\
\hspace*{0.333em} ~visit\_id = \$VISIT\_ID\\
\hspace*{0.333em} ~b = Butler(\$OUTPUT\_DATA\_DIR)\\
\hspace*{0.333em} ~b.get(``raw'', visit=visit\_id, detector=2)\\
\hspace*{0.333em} ~b.get(``bias'', visit=visit\_id, detector=2)

      \vspace{\dp0}
      } \end{minipage} \\
      \\ \cdashline{2-3}

      & Expected Result & 

      \begin{minipage}[t]{13cm}{\footnotesize
      Each call to b.get() returns an instance of an ExposureF object.
Warnings about lack of dark-time or WCS information may be ignored.

      \vspace{\dp0}
      } \end{minipage} \\
      \\ \cdashline{2-3}

      & \begin{minipage}[t]{2cm}{Actual\\ Result}\end{minipage}   & 
      \begin{minipage}[t]{13cm}{\footnotesize
      \$ python\\
Python 3.6.6 \textbar{}Anaconda, Inc.\textbar{} (default, Jun 28 2018,
17:14:51)\\
{[}GCC 7.2.0{]} on linux\\
Type ``help'', ``copyright'', ``credits'' or ``license'' for more
information.\\
\textgreater{}\textgreater{}\textgreater{} from lsst.daf.persistence
import Butler\\
\textgreater{}\textgreater{}\textgreater{} visit\_id = 258334666\\
\textgreater{}\textgreater{}\textgreater{} b =
Butler('/scratch/swinbank/ldm50307')\\
CameraMapper INFO: Loading exposure registry from
/scratch/swinbank/ldm50307/registry.sqlite3\\
CameraMapper INFO: Loading calib registry from
/scratch/swinbank/ldm50307/CALIB/calibRegistry.sqlite3\\
\textgreater{}\textgreater{}\textgreater{} b.get(``raw'',
visit=visit\_id, detector=2)\\
CameraMapper WARN: Key=``DARKTIME'' not in metadata\\
CameraMapper INFO: darkTime is NaN/Inf; using exposureTime\\
LsstCamMapper WARN: Unable to set WCS for DataId(initialdata=\{'visit':
258334666, `detector': 2\}, tag=set()) from header as
RATEL/DECTEL/ROTANGLE are unavailable\\
\textless{}lsst.afw.image.exposure.exposure.ExposureF object at
0x7ff07235ea78\textgreater{}\\
\textgreater{}\textgreater{}\textgreater{} b.get(``bias'',
visit=visit\_id, detector=2)\\
\textless{}lsst.afw.image.exposure.exposure.ExposureF object at
0x7ff09a8b88f0\textgreater{}\\
\textgreater{}\textgreater{}\textgreater{}

      \vspace{\dp0}
      } \end{minipage} \\
      \\ \cdashline{2-3}

      & Status          & Pass \\ \hline

    \end{longtable}


\input{appendix.tex}
\end{document}
