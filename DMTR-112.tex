\documentclass[DM,lsstdraft,STR,toc]{lsstdoc}
\usepackage{geometry}
\usepackage{longtable,booktabs}

\input meta.tex

\providecommand{\tightlist}{
  \setlength{\itemsep}{0pt}\setlength{\parskip}{0pt}}

\begin{document}

\def\milestoneName{Camera data processing}
\def\milestoneId{LDM-503-07}
\def\product{Data Management}

\setDocCompact{true}

\title[\milestoneId{}~Test Report]{\milestoneId{} (\milestoneName{})~Test Plan and Report}
\setDocRef{\lsstDocType-\lsstDocNum}
\setDocDate{\vcsdate}
\setDocUpstreamLocation{\url{https://github.com/lsst/lsst-texmf/examples}}
\author{ John Swinbank }

\input history_and_info.tex


\setDocAbstract{
This is the test plan and report for \milestoneId{} (\milestoneName{}), an LSST DM level 1 milestone pertaining to the \product{}.
}


\maketitle

\section{Introduction}
\label{sect:intro}


\subsection{Objectives}
\label{sect:objectives}

This test plan demonstrates that the LSST Science Pipelines can
successfully be used to load and perform basic processing on data from
the LSST Camera test systems.\\[2\baselineskip]In particular, it will
demonstrate that:\\[2\baselineskip]

\begin{itemize}
\tightlist
\item
  Data from the Camera test systems has been made available at the LSST
  Data Facility;
\item
  Data from the Camera test systems can be accessed using the ``Data
  Butler'' I/O abstraction, and loaded into the LSST Science Platform
  Notebook Aspect for processing and inspection;
\item
  Basic LSST Science Pipelines Tasks can be used to process and
  manipulate Camera data;
\item
  Camera data can be sent to the Firefly visualization tool for display.
\end{itemize}

Verification that the data processing is ``correct'' falls outside the
scope of this test plan: both Camera data and DM code is evolving
rapidly, and this exercise will not demonstrate that particular
thresholds in terms of data processing fidelity have been reached.
Rather, the focus here is on demonstrating successful integration and
interface compatibility.\\[2\baselineskip]

\subsection{Scope}\label{scope}

The overall strategy for testing and verification with LSST Data
Management is described in \citeds{LDM-503}.\\
Success in this test plan is intended to demonstrate completion of the
milestone LDM-503-07 ``Camera Data Processing''.



\subsection{System Overview}
\label{sect:systemoverview}

This test plan addresses primarily the integration between early data
coming from the LSST Camera and the data access facilities provided by
the LSST Data Management system.\\[2\baselineskip]In the process, it
will exercise:

\begin{itemize}
\tightlist
\item
  The ``Data Butler'' I/O abstraction provided by Data Management;
\item
  The Notebook Aspect of the LSST Science Platform;
\item
  Algorithmic code provided by the LSST Science Pipelines;
\item
  The Firefly image display tool provided by the Science User Interface
  and Tools group.
\end{itemize}

\textbf{Applicable Documents}\\[2\baselineskip]\citeds{LDM-503} Data Management
Test Plan\\
\citeds{LDM-151} Data Management Science Pipelines Design\\
\citeds{LDM-152} Data Management Middleware Design\\
\citeds{LDM-542} Science Platform Design


\subsection{References}
\label{sect:references}
\renewcommand{\refname}{}
\bibliography{lsst,refs,books,refs_ads}

\subsection{Document Overview and Procedure}
\label{sect:docoverview}

This document is generated from Jira, obtaining the relevant information from the 
\href{https://jira.lsstcorp.org/secure/Tests.jspa#/testPlan/LVV-P16}{LVV-P16}
~Jira Test Plan and related Test Cycles (
\href{https://jira.lsstcorp.org/secure/Tests.jspa#/testCycle/LVV-C19}{LVV-C19}
).

The following general sections are completed before the start of the test activity.

Section \ref{sect:intro} provides an overview of the test campaign, the system under test (\product{}), the documentation, and explains how this document is organized.
Section \ref{sect:configuration}  describes the configuration used for this test.
Section \ref{sect:personnel} lists all people and roles involved.
Section \ref{sect:plannedtestactivities} provides the list of planned test cycles and test cases, including all relevant information that fully describes the test campaign.
The content provided by the above sections shall be sufficient to prove that the test campaign is ready to start.

Once the above sections are completed, this document can be reviewed by the product leader, the involved personnel (section \ref{sect:personnel}), and the test campaign requester.
Once all stakeholders agree that the test pre-requisites have been satisfied, testing can begin and the Jira Test Plan shall be set to {\bf Approved} by the \product{} product leader.
A first issue of this document can be uploaded to docushare for the record.

Section \ref{sect:testresults} is filled in after the test activity is completed and all Jira Test Cycles have been set to {\bf Done}.
The first subsection \ref{sect:overview} provides a summary view of the results, in table \ref{table:summary}, 
an overall assessment statement and suggestions on possible improvements.
The subsection \ref{sect:detailedtestresults} provides detailed results for each step in each test case.

When completed, this document must be approved by the requester of the test activity and
the final issued report uploaded to docushate. 
The status of the Jira Test Plan shall then be set to {\bf Completed}.

The actual status of the Jira test plan LVV-P16 is Draft.

\section{Test Configuration}
\label{sect:configuration}

Observing is not required for this test campaign.

\subsection{Verification Environment}
\label{sect:hwconf}
Tests of the Data Butler, the Science Pipelines and the Firefly image
display tool will take place within the Notebook Aspect of the LSST
Science Platform, as deployed at
\url{https://lsst-lspdev.ncsa.illinois.edu/nb} and documented at
https://nb.lsst.io. This provides a flexible and configurable
environment with access to large-capacity filesystems at the LSST Data
Facility.\\[2\baselineskip]Individual tests will be based on specific
machine images provided within the Notebook Aspect, as documented in the
relevant test cases.~





\section{Personnel}
\label{sect:personnel}

Following personnel are involved in this test activity:

\begin{itemize}
\item Test Plan (LVV-P16) owner: John Swinbank (swinbank)
\item Test Cycles:
\begin{itemize}
  \item LVV-C19 owner: 
Undefined
  \begin{itemize}
    \item Test case LVV-T374 tester: John Swinbank (swinbank)
    \item Test case LVV-T368 tester: John Swinbank (swinbank)
  \end{itemize}
\end{itemize}
\item Additional Test Personnel involved: None
\end{itemize}

\newpage
\section{Planned Test Activities}
\label{sect:plannedtestactivities}


\subsection{Test Cycle LVV-C19}

LDM-503-07: Camera Data Processing\\
Status: Not Executed


This test cycle defines tests to be performed in late 2018 to
demonstrate the current state of integration between the Data Management
System and current Camera test datasets.



\subsubsection{LVV-T374 Test Case}

This test will check:

\begin{itemize}
\tightlist
\item
  That raw Camera test data is available on a filesystem in the LSST
  Data Facility;
\item
  That raw Camera test data can be ingested and made available through
  the Data Management I/O abstraction (the ``Data Butler'').
\end{itemize}


\begin{longtable}[]{p{1.3cm}p{15cm}}
%\toprule
Step & {Description} \\ \toprule
\endhead


\multirow{1}{*}{ 1 } &
\begin{minipage}[t]{13cm}{\footnotesize
Connect to the Notebook Aspect of the Science Platform following the
instructions at https://nb.lsst.io/. Log in, and ``spawn'' a new machine
with image ``Weekly 2018\_45`` and size ``large''.

\vspace{\dp0}
} \end{minipage} \\
\\ \midrule
\multirow{1}{*}{ 2 } &
\begin{minipage}[t]{13cm}{\footnotesize
Create a terminal session. Use it to set up the LSST tools, then
download and build version 5c12b06e6 of
obs\_lsst:\\[2\baselineskip]\hspace*{0.333em} ~\$ source
/opt/lsst/software/stack/loadLSST.bash\\
\hspace*{0.333em} ~\$ setup lsst\_distrib\\
\hspace*{0.333em} ~\$ git clone https://github.com/lsst/obs\_lsst.git\\
\hspace*{0.333em} ~\$ cd obs\_lsst\\
\hspace*{0.333em} ~\$ git checkout 5c12b06e6\\
\hspace*{0.333em} ~\$ setup -k -r .\\
\hspace*{0.333em} ~\$ scons

\vspace{\dp0}
} \end{minipage} \\
\\ \midrule
\multirow{1}{*}{ 3 } &
\begin{minipage}[t]{13cm}{\footnotesize
Ingest RTM-007 test data by executing the following
commands:\\[2\baselineskip]\hspace*{0.333em}
~OUTPUT\_REPO\_DIR=\$OUTPUT\_DATA\_DIR\\
\hspace*{0.333em} ~INPUT\_DATA\_DIR=\$INPUT\_DATA\_DIR\\
\hspace*{0.333em} ~mkdir -p \$OUTPUT\_REPO\_DIR\\
\hspace*{0.333em} ~echo ``lsst.obs.lsst.ts8.Ts8Mapper'' \textgreater{}
\$OUTPUT\_REPO\_DIR/\_mapper\\
\hspace*{0.333em} ~ingestImages.py \$OUTPUT\_REPO\_DIR
\$INPUT\_DATA\_DIR/*/*.fits\\
\hspace*{0.333em} ~constructBias.py \$OUTPUT\_REPO\_DIR --rerun calibs
--id imageType=BIAS --batch-type smp --cores 4\\
\hspace*{0.333em} ~ingestCalibs.py \$OUTPUT\_REPO\_DIR --calibType bias
\$OUTPUT\_REPO\_DIR/rerun/calibs/bias/*/*.fits --validity 9999 --output
\$OUTPUT\_REPO\_DIR/CALIB
--mode=link\\[2\baselineskip]Where:\\[2\baselineskip]\hspace*{0.333em}
~\$OUTPUT\_DATA\_DIR is some location on shared storage to which the
user has write permission;\\
\hspace*{0.333em} ~\$INPUT\_DATA\_DIR is defined in the test case
description.

\vspace{\dp0}
} \end{minipage} \\
\\ \midrule
\multirow{1}{*}{ 4 } &
\begin{minipage}[t]{13cm}{\footnotesize
Demonstrate that raw and bias data for visit \$VISIT\_ID have been made
available in the repository. Load a Python interpreter (run ``python'')
and execute the following:\\[2\baselineskip]\hspace*{0.333em} ~from
lsst.daf.persistence import Butler\\
\hspace*{0.333em} ~visit\_id = \$VISIT\_ID)\\
\hspace*{0.333em} ~b = Butler(\$OUTPUT\_DATA\_DIR)\\
\hspace*{0.333em} ~b.get(``raw'', visit=visit\_id, detector=2)\\
\hspace*{0.333em} ~b.get(``bias'', visit=visit\_id, detector=2)

\vspace{\dp0}
} \end{minipage} \\
\\ \midrule
\end{longtable}



\subsubsection{LVV-T368 Test Case}

This test will check:

\begin{itemize}
\tightlist
\item
  That Camera test data is available for processing in the LSST Data
  Facility, and accessible through the LSST Science Platform;
\item
  That the Data Management I/O abstraction (the ``Data Butler'') can
  load that data into the Science Platform environment;
\item
  That Data Management algorithmic ``tasks'' can be executed to process
  that data;
\item
  That results can be displayed in the Firefly display tool.
\end{itemize}


\begin{longtable}[]{p{1.3cm}p{15cm}}
%\toprule
Step & {Description} \\ \toprule
\endhead


\multirow{1}{*}{ 1 } &
\begin{minipage}[t]{13cm}{\footnotesize
Connect to the Notebook Aspect of the Science Platform following the
instructions at https://nb.lsst.io/. Log in, and ``spawn'' a new machine
with image ``Weekly 2018\_45`` and size ``small''.

\vspace{\dp0}
} \end{minipage} \\
\\ \midrule
\multirow{1}{*}{ 2 } &
\begin{minipage}[t]{13cm}{\footnotesize
Create a terminal session. Use it to set up the LSST tools, then
download and build version 5c12b06e6 of
obs\_lsst:\\[2\baselineskip]\hspace*{0.333em} ~\$ source
/opt/lsst/software/stack/loadLSST.bash\\
\hspace*{0.333em} ~\$ setup lsst\_distrib\\
\hspace*{0.333em} ~\$ git clone https://github.com/lsst/obs\_lsst.git\\
\hspace*{0.333em} ~\$ cd obs\_lsst\\
\hspace*{0.333em} ~\$ git checkout 5c12b06e6\\
\hspace*{0.333em} ~\$ setup -k -r .\\
\hspace*{0.333em} ~\$ scons\\[2\baselineskip]Arrange for obs\_lsst to
automatically be added to the environment when starting a new
notebook:\\[2\baselineskip]\hspace*{0.333em} ~\$ echo ``setup -j -r
\textasciitilde{}/obs\_lsst'' \textgreater{}\textgreater{}
notebooks/.user\_setups\\[2\baselineskip]Exit the terminal.

\vspace{\dp0}
} \end{minipage} \\
\\ \midrule
\multirow{1}{*}{ 3 } &
\begin{minipage}[t]{13cm}{\footnotesize
Create a new ``LSST'' notebook.\\[2\baselineskip]Import the standard
libraries required for the rest of this
test:\\[2\baselineskip]\hspace*{0.333em} ~import os\\
\hspace*{0.333em} ~import lsst.afw.display as afwDisplay\\
\hspace*{0.333em} ~from lsst.daf.persistence import Butler\\
\hspace*{0.333em} ~from lsst.ip.isr import IsrTask\\
\hspace*{0.333em} ~from firefly\_client import FireflyClient\\
\hspace*{0.333em} ~from IPython.display import
IFrame\\[2\baselineskip]and execute the cell.

\vspace{\dp0}
} \end{minipage} \\
\\ \midrule
\multirow{1}{*}{ 4 } &
\begin{minipage}[t]{13cm}{\footnotesize
Create a Data Butler client, and use it to retrieve the data which will
be used for this test.\\[2\baselineskip]\hspace*{0.333em} ~butler =
Butler(\$REPOSITORY\_PATH)\\
\hspace*{0.333em} ~raw = butler.get(``raw'', visit=\$VISIT\_ID,
detector=2)\\
\hspace*{0.333em} ~bias = butler.get(``bias'', visit=\$VISIT\_ID,
detector=2)

\vspace{\dp0}
} \end{minipage} \\
\\ \midrule
\multirow{1}{*}{ 5 } &
\begin{minipage}[t]{13cm}{\footnotesize
Initialize the Firefly display
system:\\[2\baselineskip]\hspace*{0.333em} ~my\_channel =
`\{\}\_test\_channel'.format(os.environ{[}'USER'{]})\\
\hspace*{0.333em} ~server = `https://lsst-lspdev.ncsa.illinois.edu'\\
\hspace*{0.333em}
~ff='\{\}/firefly/slate.html?\_\_wsch=\{\}'.format(server,
my\_channel)\\
\hspace*{0.333em} ~IFrame(ff,800,600)\\
\hspace*{0.333em} ~afwDisplay.setDefaultBackend('firefly')\\
\hspace*{0.333em} ~afw\_display = afwDisplay.getDisplay(frame=1,\\
\hspace*{0.333em} ~ ~ ~ ~ ~ ~ ~ ~ ~ ~ ~ ~ ~ ~ ~ ~ ~ ~
~name=my\_channel)\\[2\baselineskip]Click on the link provided after
executing the above.

\vspace{\dp0}
} \end{minipage} \\
\\ \midrule
\multirow{1}{*}{ 6 } &
\begin{minipage}[t]{13cm}{\footnotesize
Display the raw image data in the Firefly
window:\\[2\baselineskip]\hspace*{0.333em} afw\_display.mtv(raw)

\vspace{\dp0}
} \end{minipage} \\
\\ \midrule
\multirow{1}{*}{ 7 } &
\begin{minipage}[t]{13cm}{\footnotesize
Configure and run an Instrument Signature Removal (ISR) task on the raw
data. Most corrections are disabled for simplicity. but the bias frame
is applied.\\
\hspace*{0.333em}

\begin{verbatim}
   isr_config = IsrTask.ConfigClass()
   isr_config.doDark=False
   isr_config.doFlat=False
   isr_config.doFringe=False
   isr_config.doDefect=False
   isr_config.doAddDistortionModel=False
   isr_config.doLinearize=False
   isr = IsrTask(config=isr_config)
   result = isr.run(raw, bias=bias)
\end{verbatim}

\vspace{\dp0}
} \end{minipage} \\
\\ \midrule
\multirow{1}{*}{ 8 } &
\begin{minipage}[t]{13cm}{\footnotesize
Display the corrected image data in the Firefly
window:\\[2\baselineskip]\hspace*{0.333em}
~afw\_display.mtv(result.exposure)

\vspace{\dp0}
} \end{minipage} \\
\\ \midrule
\end{longtable}




\newpage

\section{Test Results}
\label{sect:testresults}

\subsection{Overview of the Test Results}
\label{sect:overview}

\subsubsection{Summary Table}
\label{sect:summarytable}


\begin{longtable} {p{0.2\textwidth}p{0.2\textwidth}p{0.6\textwidth}}
\toprule
\multicolumn{3}{c}{ Test Cycle {\bf LVV-C19: LDM-503-07: Camera Data Processing }} \\\hline
{\bf \footnotesize test case id} & {\bf \footnotesize status} & {\bf \footnotesize comment} \\\toprule
LVV-T374 & Not Executed &  \\\hline
LVV-T368 & Not Executed &  \\\hline
\caption{Test Results Summary Table}
\label{table:summary}
\end{longtable}

\subsubsection{Overall Assessment}
\label{sect:overallassessment}


\subsubsection{Recommended Improvements}
\label{sect:recommendations}

\subsection{Detailed Test Results}
\label{sect:detailedtestresults}

\end{document}
